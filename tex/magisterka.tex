\documentclass{xmgr}

\makeatletter
\newcommand\footnoteref[1]{\protected@xdef\@thefnmark{\ref{#1}}\@footnotemark}
\makeatother

\author   {Daniel Sienkiewicz}
\nralbumu {206358}
\email    {daniel@sienkiewicz.ovh}


\title    {Projekt komputera samochodowego bazujący na systemie mikrokomputera Intel Galileo}
\date     {2015}
\miejsce  {Gdańsk}

\opiekun  {dr inż. Janusz Młodzianowski}

\begin{document}

\begin{abstract}
Celem pracy jest stworzenie komputera pokładowego do samochodu w którego skład wchodzi: \begin{enumerate}
\item Mikrokomputer Intel Galileo Gen 1, 
\item Symulator samochodu, 
\item Ekran dotykowy FTDI VM800, 
\item Oprogramowanie.
\end{enumerate}

Ponadto praca zawiera propozycje dalszego rozwoju projektu. Przedstawione zostały możliwości dodania dodatkowych modułów w celu zwiększenia funkcjonalności oraz wykorzystania do nowych zadań.
\end{abstract}
\keywords{Intel Galileo, $I^2C$,
 SPI, 
 C, 
 Arduino,
 GPIO}
\maketitle

%================WPROWADZENIE=====================
\chapter{Wprowadzenie}
TO DO
\section{Cele}
TO DO
\section{Założenia}
- galileo + arduino (środowisko)
- dostepne czujniki
- kamerka - czujnik cofania /lusterko wsteczne
- GPS - rejestracja czasu drogi
- komputer nie do samochodu tylko SYMULATOR!!!
- immo???
- nie ma predkosci i przebiegu!!! bo NIE
\section{Plan pracy}
TO DO
%================KONIEC WPROWADZENIE=====================

%================ARCHITEKTURA=====================
\chapter{Architektura}
\subsection{Schemat blokowy i opis funkcjonalny}
\subsection{Mechanizmy komunikacji systemu mikroprocesorowego z otoczeniem}
- porty
- przerwania
- dopytywanie
- timer
- protokół komunikacyjny (co to jest) np I2C
- złączki, kable itp
// po 1 stronie max na kazdy

%================KONIEC ARCHITEKTURA=====================

%================IMPLEMENTACJA=====================
\chapter{Implementacja}
\section{I2C}
\subsection{Problemy z bibliotekami}
\subsection{moja implenatacja I2C (read)}
\subsection{Schemat blokowy programu}
\subsection{Moja biblioteka do R/W Arduino galileo - COŚ CZEGO NIE MA JESZCZE NIGDZIE!!!!}
\section{Założenia funkcjonalne}
 - czytanie z czyjników, pisanie do ekaranu, zczytanie z ekranu
 - schemat blokowy z BAJERAMI i wybrane to co zrobie
 - włączanie i wyłączanie systemu
\section{Integracja z samochodem}
 - podpięcie pod auto
 - włączanie i wyłączanie systemu - mozna brutalnie wylaczyc
\section{VM800}
  - na poczatku emulacja na PC
  - potem przepisanie na niski poziom
  - ostatecznie podpiecie do Galielo (poszukac czy juz jest?)
  
   programers manual reference vm800 ftdi POSZUKAĆ!!!!
%================KONIEC IMPLEMENTACJA=====================

\summary
TO DO

\appendix
\chapter{Programy}

\bibliographystyle{unsrt}
\bibliography{xml}

\listoftables

\listoffigures

\oswiadczenie

\end{document}
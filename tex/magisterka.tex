\documentclass{xmgr}

\makeatletter
\newcommand\footnoteref[1]{\protected@xdef\@thefnmark{\ref{#1}}\@footnotemark}
\makeatother

\author   {Daniel Sienkiewicz}
\nralbumu {206358}
\email    {daniel@sienkiewicz.ovh}


\title    {Projekt komputera samochodowego bazujący na systemie mikrokomputera Intel Galileo}
\date     {2015}
\miejsce  {Gdańsk}

\opiekun  {dr Janusz Młodzianowski}

\begin{document}

\begin{abstract}
Celem pracy jest stworzenie komputera pokładowego do samochodu w którego skład wchodzi: \begin{enumerate}
\item Mikrokomputer Intel Galileo Gen 1, 
\item Symulator samochodu, 
\item Ekran dotykowy FTDI VM800, 
\item Oprogramowanie.
\end{enumerate}

Ponadto praca zawiera propozycje dalszego rozwoju projektu. Przedstawione zostały możliwości dodania dodatkowych modułów w celu zwiększenia funkcjonalności oraz wykorzystania do nowych zadań.
\end{abstract}
\keywords{Intel Galileo, $I^2C$,
 SPI, 
 C, 
 Arduino,
 GPIO}
\maketitle

%================WPROWADZENIE=====================
\chapter{Wprowadzenie}
TO DO
\section{Cele}
TO DO
\section{Założenia}
TO DO
\section{Plan pracy}
TO DO
%================KONIEC WPROWADZENIE=====================

%================ARCHITEKTURA=====================
\chapter{Architektura}
\section{Opis wersji, etapy pracy nad sprzętem}
\subsection{Porównanie dostępnych na rynku mikrokomputerów}
TO DO
\section{Obsługa urządzeń wejścia/wyjścia w różnych systemach operacyjnych}
TO DO
\subsection{Podstawowe interfejsy I/O}
TO DO
\subsubsection{SPI}
TO DO
\subsubsection{$I^2C$}
TO DO
\subsubsection{USB OTG}
TO DO
\subsubsection{GPS}
TO DO
\subsubsection{Wyjścia analogowe i cyfrowe}
TO DO
\subsection{Symulator samochodu}
TO DO
%================KONIEC ARCHITEKTURA=====================

%================IMPLEMENTACJA=====================
\chapter{Implementacja}
\section{Wizja programu}
TO DO
\section{Opis funkcji}
\subsection{Schemat blokowy programu}
TO DO
\section{Użyte algorytmy}
\subsection{Próbkowanie sygnału}
TO DO
\section{Schematy sprzętu}
TO DO
%================KONIEC IMPLEMENTACJA=====================

\summary
TO DO

\appendix
\chapter{Programy}

\bibliographystyle{unsrt}
\bibliography{xml}

\listoftables

\listoffigures

\oswiadczenie

\end{document}